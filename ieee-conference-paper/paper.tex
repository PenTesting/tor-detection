% IEEE Conference paper document class
\documentclass[conference]{IEEEtran}

% Use hyperlinks for convenience, but let's not make them coloured
\RequirePackage[bookmarks,colorlinks,breaklinks]{hyperref}
\hypersetup{
  colorlinks,
  citecolor=black,
  filecolor=black,
  linkcolor=black,
  urlcolor=black
}

% Use biblatex with IEEE style
\RequirePackage[american]{babel}
\RequirePackage{csquotes}

% Use the excellent biblatex package with apa style
% minnames=1,maxnames=2 - abbreviates all author names in citations
% hyperref - links in pdf
\RequirePackage[
  backend=biber,
  url = false,
  hyperref = true,
  style = ieee,
  minnames = 1,
  maxnames = 2,
  bibencoding = utf8]{biblatex}
\DeclareLanguageMapping{american}{american-apa}
\DefineBibliographyStrings{american}{% adapt language
  page = {{}p\adddot},
  pages = {{}pp\adddot},
}

% Bibliography
\bibliography{fullrefs}

\begin{document}

\title{Using traffic analysis to identify The Second Generation Onion Router}

\author{
  \IEEEauthorblockN{John Barker}
  \IEEEauthorblockA{
  School of Computer and\\Security Science\\
  Edith Cowan University\\
  Mt Lawley, Western Australia\\
  Email: john.barker@our.ecu.edu.au}
  \and
  \IEEEauthorblockN{Peter Hannay}
  \IEEEauthorblockA{
  School of Computer and\\Security Science\\
  Edith Cowan University\\
  Mt Lawley, Western Australia\\
  Email: p.hannay@ecu.edu.au\\
  Telephone: (08) 9370 6374}
  \and
  \IEEEauthorblockN{Patryk Szewczyk}
  \IEEEauthorblockA{
  School of Computer and\\Security Science\\
  Edith Cowan University\\
  Mt Lawley, Western Australia\\
  Email: p.szewczyk@ecu.edu.au\\
  Telephone: (08) 9370 6979\\
  Fax: (08) 9370 6100}
}

\maketitle

\begin{abstract}
Anonymous networks provide security for users by obfuscating messages with
encryption and hiding communications amongst cover traffic provided by other
network participants. The traditional goal of academic research into these
networks has been attacks that aim to uncover the identity of network users.
But the success of an anonymous network relies not only on it's technical
capabilities, but on adoption by a large enough user base to provide adequate
cover traffic. If anonymous network nodes can be identified, the users
can be harassed, discouraging participation. Tor is an example of widely used
anonymous network which uses a form of Onion Routing to provide low latency
anonymous communications. This thesis demonstrates that traffic from a simulated
Tor network can be distinguished from regular encrypted traffic, suggesting that
real world Tor users may be vulnerable to the same analysis.
\end{abstract}

\section{Introduction}

The first anonymous digital network, commonly known as MixNet was proposed by
\citeauthor{Chaum:1981p296} in \citetitle{Chaum:1981p296}
\parencite{Chaum:1981p296}. This paper introduced a concept integral to many
future anonymity providing designs, an intermediate system responsible for
delivering messages without the identifying details of correspondents. The
intermediate system, referred to as a 'mix' also employed public key
cryptography to ensure that eavesdroppers could not obtain delivery information.

This seminal paper spurred research into new techniques for providing anonymity
and privacy for digital networks. One of these, The Second Generation Onion
Router (Tor) is based on technology originally designed by the U.S. Naval
Research Lab in 1996 \parencite{Goldschlag:1996wy} and enjoys some measure of
popularity, with an average of two hundred thousand active users as of March
2011 \parencite{The-Tor-Project-Inc.:2011fk}.

\section{The Second Generation Onion Router (Tor)}

Like a mix, messages sent over an onion routing network were encrypted with
their routing information and delivered to an intermediate server for forward
delivery.  Unlike the mix however, messages delivered using the onion routing
network were encrypted multiple times, each 'layer' using a different key and
routing instructions. The first node in a chain would only be able to encrypt
the routing instructions to deliver the message to the next node. Each node in
the sequence decrypting a layer until the complete message is decrypted and
transmitted to the destination.

Traffic enters the Tor network through an onion proxy which accepts TCP streams.
Some identifying features are scrubbed from the data using application filters
before the data is relayed over the network through TLS \parencite{website:TLS}
encrypted connections. The intermediate nodes responsible for routing messages
are known as relays and are typically chained together to construct a circuit.
When traffic leaves the Tor network it is delivered by a special kind of relay
known as an exit node. At an exit node the data is transmitted in the original
format it was supplied to at the onion proxy.

The onion proxy builds circuits incrementally, first obtaining a session key
from each successive relay in a circuit. Once all session keys for a circuit
have been obtained, the message is broken into fixed sized cells of 512 bytes
and iteratively encrypted using the session key of each node in the circuit in
the reverse order that the data traverses the network. Cells come in two forms:
control cells and relay cells. Control cells are used to create and maintain
circuits, while relay cells contain commands for circuit maintenance and
additional data for verifying message integrity and identifying streams.

When the first node in the circuit receives a cell, it removes a layer by
decrypting the message using it's session key. The cell is delivered along the
circuit and at each node a layer is removed until the cell is completely
decrypted and can be delivered successfully. Since circuits can take some time
to build, they are pre-emptively created at regular intervals

\section{Weaknesses and Attacks}

When designing a system such as Tor, a number of trade-offs have to be made
between the strength of the security provided and the convenience and
performance of the system. The Tor designers consciously prioritized low
latency, usability and flexibility against security goals such as security
against end to end attacks \parencite[4]{Dingledine:2004p314}. These design
decisions mean that Tor is vulnerable to several kinds of attacks, some of them
expected and some not.

A well known attack involves sniffing traffic that leaves exit nodes to
capture private information \parencite{website:tor-password-leak}. Many users
assume that Tor provides end to end encryption, and transmit private information
over the Tor network.

Most conventional attacks against secure networks are known as traffic analysis,
this is the process of examining information about the communications rather
than the information contained within them. This information may include the
size of messages communicated, their frequency and timing. Many researchers have
proposed traffic analysis techniques that allow attackers to reveal the
identities of Tor users.

Technologies such as Javascript and Flash can be embedded in web pages accessed
by Tor users, and have control over network resources. By injecting network
traffic with certain patterns alongside regular network traffic, Tor users can
be identified \parencite{Abbott:2007p298}.

Tor bridges, intended as a way to get censored users access to Tor are easily
identified. They are also vulnerable to clogging attacks which make bridge
operators more easily identified \textcite{McLachlan:2009p197}.

\textcite{Murdoch:2005p325} proposes a technique to identify users by estimating
the latency of individual Tor nodes using a hostile Tor node. The hostile Tor
node is able to send data to users using a predictable traffic pattern and
identify this pattern as it is repeated through the network, correlating streams
back to individuals.

This attack was shown to be impractical as the Tor network grew in size, with
the increased number of users adding enough congestion to mask the introduced
identifying patterns \parencite{Evans:2009p315}. However a weakness in the Tor
design meant that particularly configured hostile nodes could amplify the
deliberately introduced congestion to make individuals identifiable on the
larger network.

\section{Significance}

While Tor has been the subject of much academic literature, the primary
objective of researchers has historically been the attempt to reveal participant
identities \parencite[3]{Murdoch:2005p325}.  Many attack techniques proposed
have been somewhat academic in nature and not necessarily feasible in practice
\parencite{Raccoon:2008fk}. Sometimes they require compromise of large parts of
the Tor network, supplying hostile data to Tor users or complicated knowledge of
usage patterns and an excess of patience. The technique demonstrated in this
thesis is a low cost technique, which does not require very sophisticated
equipment and can be completed by a passive observer.

\parencite[5]{Dingledine:2004p314}.

\printbibliography

\end{document}
