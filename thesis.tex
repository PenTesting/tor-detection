\abstract The Internet is a powerful tool for publishing speech and opinion, for this reason it has significant implications for policial speech. Unfortunately this ability to speak out is being destroyed by an increasingly hungry powerful elite who attempt to stomp out political assent by any means necessary. Dissenters who publish their opinions using the internet are increasingly being identified and subject to imprisonment. Tools are available to hide provide levels of anonymity to such users, but these tools really provide anonymity? Can they provide anonymity in such regimes if they are easily distinguishable from normal traffic? This paper seeks to investigate the qualities of encrypted communications on the Internet and compare normal encrypted communications to those performed by anonymizing tools such as TOR and I2P. A number of simulations will be run with these applications and the traffic monitored to gather important information and characteristics. These will be compared to identify what characteristics make a particular application identifiable and what qualities are used for this identity. The aim of this research is to provide insight into what qualities other than data make an application uniquely identifiable, despite using strong encryption.


